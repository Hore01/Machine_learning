% ------------------------------------------------------------------------
% -*-TeX-*- -*-Hard-*- Smart Wrapping
% ------------------------------------------------------------------------
\def\baselinestretch{1}

\chapter{Literature Review}

\def\baselinestretch{1.44}

%%% ----------------------------------------------------------------------

This chapter reviews the related literature to my dissertation topic. 
   
\smallskip

%%% ----------------------------------------------------------------------
\goodbreak
\section{Review on basic methods}
Machine learning (ML) has significantly advanced various fields, including “computer vision, natural language processing(NLT), artificial intelligence, and speech recognition”. It allows developers and researchers to extract essential insights from datasets, deliver tailored experiences, and create advanced systems \citep{jordan2015machine}. Machine learning has facilitated rapid and extensive analysis of intricate data in areas such as bioinformatics \citep{luo2016big}. These approaches are now being implemented in mental health data, offering substantial potential to improve patient outcomes and enhance the comprehension of psychological disorders and their treatment \citep{shatte2019machine}.

\citet{jordan2015machine}Shivangi(2015) investigated the use of various ML algorithms to predict depression using routine survey data, examining the methodologies and outcomes to provide insights into the efficacy of these approaches. The study utilized routine survey data, which included questions about home and workplace environments, family history of mental illness, and personal health information. A variety of machine learning algorithms were used to determine the most effective method for predicting depression, which include: K-Nearest Neighbors (KNN), Boosting, Decision trees, Random forest, Logistic regression, and Bagging (Bootstrap Aggregating). The results demonstrated that ensemble methods, particularly Boosting with  accuracy of 81.75\%, and Random Forest with accuracy of 81.22\%, significantly outperformed other algorithms in predicting depression.

In addition, \citet{sau2017predicting} investigated the use of machine learning classifiers to predict depression in elderly citizens, aiming for early diagnosis and improved medical intervention using the WEKA data mining tool.  Data were collected from senior citizens in Kolkata.Geriatric Depression Scale and classifiers such as Logistic Regression, Bayes Net, Multilayer Perceptron (MLP), Decision Tree and  Sequential Minimal Optimization (SMO), were utilized for the prediction. The study found that Bayes Net provided the best overall performance in predicting depression, particularly when using the percentage split test option in WEKA. This approach promises to enhance early detection and treatment of depression among seniors, reducing the manual burden on healthcare providers. 

\citet{malik2023anxiety} explores the application of machine learning techniques to predict the severity levels of anxiety, depression, and stress among college students. The study used the DASS21 questionnaire to collect data from 400 students, classifying their responses into five severity levels: “normal, mild, moderate, severe, and extremely severe”. Machine learning algorithms, including Support Vector Machine (SVM), Decision tree, K-Nearest Neighbor (KNN), Naïve Bayes, and logistic regression were used to predict these mental health issues. Performance metrics such as F1 score, accuracy, recall, precision and specificity were used to evaluate the models, with a focus on addressing class imbalance. KNN emerged as the best-performing model with the highest accuracy and F1 score, making it the most effective in handling imbalanced data and predicting mental health severity accurately.

 \citet{haque2021detection} however studied the use of machine learning to detect depression in children and adolescents. Recognizing the profound impact of childhood depression on individuals, families, and society, the study sought to create an early detection model. The research utilized the Young Minds Matter (YMM) dataset involving 6,310 participants, which includes comprehensive data on Australian children aged 4-17. For the feature selection, the Boruta algorithm was combined with a Random Forest (RF) classifier. Eleven critical features were identified for detecting depression: unhappiness, sleep disturbance, suicidal thoughts, lack of fun, weight changes, irritable mood, fatigue, diminished interest, psychomotor changes, concentration issues, and the presence of any five of these symptoms. Tree-based Pipeline Optimization Tool (TPOT classifier) was used to select and optimize the tested models which included: Decision Tree (DT), Random Forest (RF), Gaussian Naive Bayes (GaussianNB), and XGBoost (XGB).  Among the models, Random Forest (RF) outperformed others, achieving 99\% precision and 95\% accuracy, with a runtime of 315 milliseconds. The study highlights the significance of parental and social involvement in a child's life and the adverse effects of family disruptions and poor health conditions on mental health.

Another notable mental health illness that machine learning helps to predict is Alzheimer. \citet{salunkhe2022prediction} in their study used Random Forests (RF) and Support Vector Machines (SVM) to develop predictive models for diagnosing Alzheimer's using the OASIS dataset. The OASIS dataset, containing MRI scans and other attributes, was used and it was observed that SVM achieved higher accuracy (93\%), precision, recall, and F1 score compared to RF. 


Suicide is also a major public health issue affecting millions globally, especially the youth. \citet{faisal2023machine} studied using machine learning (ML) to predict suicidal thoughts and behaviors. The dataset used in the research work was obtained from the “Global School-based Student Health Survey (GSHS)”  which includes data from teenagers across 26 countries. The data includes various attributes such as alcohol consumption, parental understanding, drug use, past suicide attempts and bullying experiences.  Machine learning models used in the research work include Support Vector Regression (SVR), Random Forest (RF), Linear Regression (LR), Ridge Regression (RR), K-Nearest Neighbors (KNN), Multi-Layer Perceptron (MLP), and Extreme Gradient Boosting (XGB). SVR was found to have the best performance with the lowest MAE (4.6588), MAPE (0.6561) and RMSE (4.6589), indicating it is the most suitable model for predicting suicidal behavior by analyzing various risk factors. The study highlights the potential of Machine Learning to significantly aid in suicide prevention efforts by providing early detection and intervention.

In his study, \citet{talaei2019predictive} investigated predicting multiple medical complications among patients with chronic diseases using predictive analytics. The data was mined from the Electronic medical records (EMRs) of 1078 patients with hypertrophic cardiomyopathy from 2009 to 2017. Models such as Support vector machines (SVM),  Decision trees (DT), Logistic regression (LR), and artificial neural networks (ANN) were used in predicting complications such as Heart failure, cardiac arrest and problems with heart valves. Independent Prediction of Multiple Complications (IPMC) involves the use of single-task learning (STL) to predict each complication independently whereas, Concurrent Prediction of Multiple Complications (CPMC) was used to predict complications concurrently, considering their interrelationships. The study compares these two methods and demonstrates that CPMC accounts for the interrelationships among complications.  Discrimination (AUC) and calibration (Q-Score) performance metrics were used and it was observed that CPMC outperforms in both discrimination, meaning its capability to differentiate between patients who will develop complications and those who will not, and calibration (the accuracy of the predicted probabilities). 
\citet{abdillah2021comparative} compared single-task learning (SLT) and Multi-task Learning(MLT) in the context of research protocol classification, particularly in the health domain, to aid the ethical review process by the ethical committee. The study utilized two datasets: one with 36 documents related to ethical protocols and another with 20,972 research paper summaries. The documents were processed and annotated for multiple labels indicating ethical issues and research topics. Although , the  MTL outperformed STL in terms of Hamming Loss (0.125 vs. 0.182 on the smaller dataset and 0.097 vs. 0.113 on the larger dataset) and Jaccard Score (0.785 vs. 0.720 on the smaller dataset and 0.635 vs. 0.571 on the larger dataset), but in terms of computation time, MTL was slower, taking 27\% more time on the smaller dataset and 41\% more time on the larger dataset. It was concluded that despite higher computational costs, MTL showed better performance in multi-label classification tasks. 

\section{Related Work}
\subsection{Single-task Learning}  
The Cross Check study represents a significant step toward the development of passive sensing systems for mental health monitoring. It underscores the potential of leveraging smartphone technology to enhance the care and management of individuals with schizophrenia through early detection and intervention strategies. \citet{wang2016crosscheck}  initiated the first attempt to address the pressing need for early prediction of mental health changes in individuals diagnosed with severe mental illness, with a particular emphasis on those suffering from schizophrenia. 21 outpatients with schizophrenia were monitored after hospital discharge over a period of 2-8.5 months. Smartphones collected passive sensor data, including information on sleep, phone usage, conversations and mobility and participants completed periodic self-reports on their mental health. The study analyzed the correlation between passive sensor data and self-reported Ecological Momentary Assessment(EMA). Inference models were built to predict mental health scores with a mean error of 7.6\%.  Statistically significant associations were found between passive behavioral features and self-reported EMA.The study demonstrated that models could predict mental health scores accurately by leveraging data from the population and adapting to individual-specific data. Personalized models required fewer data points to adapt quickly to new users.  The predictive models were able to achieve a mean absolute error (MAE) of 1.378 for positive scores and 1.383 for negative scores, with strong correlations (r values) between predicted and actual scores. 

In 2017,\citet{wang2017predicting} advanced the research by utilizing mobile phone sensor data to predict the severity of symptoms as assessed by the Brief Psychiatric Rating Scale (BPRS). The system predicts weekly scores on a 7-item Brief Psychiatric Rating Scale (BPRS), assessing symptoms like grandiosity, hallucinations and more. The study compares the predictive accuracy of models using passive sensing, EMA and a combination of both. The best results achieved a mean absolute error (MAE) of 1.45, showing a strong correlation between predicted and actual BPRS scores. The system identifies at-risk patients based on score thresholds and trends, enabling timely clinical interventions. The research highlights the potential of mobile sensing for continuous symptom monitoring, although it notes limitations such as an unbalanced dataset and the need for more data from patients with severe symptoms. Future improvements include better training models with additional high-severity data and addressing generalization to different environments. However, \citet{wang2020predicting} study addresses challenges in predicting rare events in a longitudinal dataset and finds the best prediction results using principal components (PCA) from both passive sensing and self-reports with 30-day prediction windows. 

In their research, \cite{adler2020predicting} utilized two distinct encoder-decoder neural network architectures in combination with a clustering-based local outlier factor model to forecast behavioral deviations during the 30-day period leading up to a relapse, referred to as the near relapse period. The models were trained to replicate behaviors representative of "days of relative health" (DRH), which occur outside this near relapse window. Following this, a post hoc analysis was conducted using the most effective model to identify behavioral traits that significantly contributed to distinguishing anomalies near relapse from DRH. This analysis focused on four participants who experienced multiple relapses during the study. The most successful model, a fully connected neural network autoencoder, demonstrated a median sensitivity of 0.25 (interquartile range: 0.15-1.00) and a specificity of 0.88 (interquartile range: 0.14-0.96), indicating a 108\% improvement in the median detection of behavioral anomalies near relapse. The proposed approach was more effective in predicting anomalies among patients with Serious Mental Health Disorders (SSDs) in the 30 days preceding relapse and can be applied to identify behavioral changes leading up to relapse events. 

However, \citet{lamichhane2022psychotic}introduces RelapsePredNet, a personalized Long Short-Term Memory (LSTM) neural network-based model for predicting psychotic relapse in schizophrenia patients using mobile sensing data. Leveraging continuous mobile sensing data from 63 schizophrenia patients in the CrossCheck dataset, the model demonstrated substantial enhancements in prediction accuracy compared to current approaches. The personalized RelapsePredNet model outperformed non-personalized versions and existing anomaly detection models. Personalization was crucial for improving prediction accuracy, and the study opens avenues for further research into personalized and real-time monitoring systems for mental health management.
\subsection{Multi- task Learning}
Previous research by \citet{adler2020predicting} focused on data from single studies with homogeneous populations and short periods, limiting model applicability to broader contexts. In a recent study, \citet{adler2022machine} merged data from the CrossCheck study, involving schizophrenia patients, and the StudentLife study, focused on university students, to evaluate whether models trained on this combined dataset offer better prediction accuracy than those trained on data from just one of the studies. Models trained with combined data from both studies showed better prediction performance than those trained on single-study data. Gradient Boosting Regression Trees (GBRT) were utilized to forecast mental health symptoms, while hyperparameters such as the learning rate, the number of trees, and the depth of the trees were modified. In addition, personalization and oversampling techniques were applied to enhance model accuracy. The research shows that machine learning models, when trained on integrated data from various diverse studies, are capable of effectively generalizing across different populations and settings.

Moreover, \citet{tseng2020using} explores the prediction of schizophrenia symptom progression through the application of "behavioral rhythms and multi-task learning (MTL) models." The research employed MTL to forecast ecological momentary assessment scores across ten distinct symptoms. Results indicated that MTL models outperformed single-task learning models in accurately predicting the trajectories of individual symptoms, including emotional states such as depression, auditory hallucinations, sociability, and calmness. In addition, the models enhanced interpretability by accounting for participant similarities and differences in symptom manifestation. Unsupervised clustering further identified unique subtypes within each symptom category based on feature weights. These findings have substantial clinical implications, supporting early detection and intervention strategies without imposing additional burdens on patients or clinicians.

In conclusion, the literature reviewed consistently supports the notion that multi-task learning (MTL) outperforms single-task learning (STL) in various applications, particularly in the context of predicting complex symptom trajectories. This alignment across studies underscores the robustness and generalizability of MTL models in handling interconnected tasks, such as the prediction of multiple schizophrenia symptoms. The collective evidence highlights that MTL not only enhances predictive accuracy but also improves the interpretability of the models by accounting for the relationships between different tasks. This consensus among the literature directly aligns with the research objectives of this work, which aims to investigate the superior performance of MTL over STL for predicting EMA score. The findings of this study will contribute to this body of knowledge by using a model that has both variants of SLT and MLT to verify whether MTL models will provide a more effective approach for forecasting individual Schizophrenia symptom progressions, ultimately supporting the advancement of more nuanced and targeted clinical interventions.

\section{Summary} 
The review of literature in this chapter has revealed the consensus between different study that multi-task learning outperforms single-task learning for the prediction of various health-related conditions such as Depression, Alzheimer, Suicide and also Schizophrenia. In multi-task learning, the model is trained simultaneously by sharing of information across multiple related tasks, whereas in single-task learning, each task is trained independently and without interaction, thus not benefiting from shared information. The findings demonstrate that the sharing of information between multiple related tasks in multi-task learning significantly enhances the overall generalization performance of the model. Additionally, multi-task learning has proven to be more resource-efficient, as it enables the simultaneous training of multiple tasks, thereby eliminating the necessity for individual models for each task. However, it is important to acknowledge that despite its efficiency in resource utilization, multi-task learning can be more computationally expensive due to the complexity involved in managing and optimizing several tasks concurrently. This added complexity may require more sophisticated infrastructure and extended training times, which could be considered a trade-off for the enhanced performance and interpretability that MTL provides. This advantage is particularly pronounced in the context of mental health conditions such as schizophrenia. Multi-task learning enables the integration of various related tasks, including symptom prediction, treatment response, and patient outcome monitoring, into a unified and comprehensive model. By capitalizing on the interconnected nature of these tasks, multi-task learning can significantly enhance predictive accuracy and treatment effectiveness, offering a more holistic approach to managing and understanding schizophrenia. Overall, this chapter highlights the superior performance of multi-task learning compared to single-task learning, emphasizing its broader applicability in diverse machine learning scenarios, with a special focus on its effectiveness in mental health applications like schizophrenia.

   


\def\baselinestretch{1.66}
\medskip


%%% ----------------------------------------------------------------------
